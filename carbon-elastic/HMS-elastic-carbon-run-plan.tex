\documentclass[]{article}

%opening
\title{Carbon elastics for HMS commissioning at 1 pass}

\begin{document}

\maketitle


\section{Introduction}

The first excited state of carbon elastics will be used to test the HMS optics.
Important to following the HMS cycling procedure.
The running conditions:
\begin{table}[h]
	\begin{center}
		\begin{tabular}[]{|c|c|} \hline\hline
			beam energy: & 2218 MeV\\ \hline
			HMS momentum & 2201.54 MeV \\ \hline
			HMS angle & 13.5 deg \\ \hline
			HMS collimator & Sieve \\ \hline
			beam current: & 1 μA\\ \hline
			fast raster: & off\\ \hline
			target: & 0.5\% carbon\\ \hline
		\end{tabular}
		\caption{Kinematics}
	\end{center}
\end{table}

\section{Surveys}

The surveys for the HMS are given in the Table.  Pointing is given in the spectrometer 
coordinate system with +X downwards and +Y towards smaller angles.
	\begin{table}[h]
		\begin{center}
			\begin{tabular}[]{|c|c||c|c|} \hline\hline
				Survey & Angle  & Horizontal point (Y$_{spec}$) & Vertical point (X$_{spec}$)\\ \hline
				C1792R & 40.534 & +3.45& +0.91\\ \hline
				C1807R & 40.013 & +3.29 & +0.94 \\ \hline
				C1809R & 50.00  & +2.92 & +1.07 \\ \hline
			\end{tabular}
			\caption{Survey are the HMS}
		\end{center}
	\end{table}
	
	\section{First order optics}
	HMS first order forward optics:
\begin{eqnarray}
	xfp (mm) &=& -3.41*xtar (mm) - 0.02*xptar (mr) +37.0*delta \\
	xpfp (mr) &=& -.04*xtar (mm) - .29*xptar (mr) - 0.4*delta \\
	yfp (mm) &=& -2.2*ytar (mr) - 0.02*yptar (mr)  \\
	ypfp (mr) &=& -2.62*ytar (mm) - 0.42*yptar (mr) 
\end{eqnarray}
	
	HMS first order reconstruction optics.
\begin{eqnarray}
	xptar (mr) = 0.34 xfp ( mm) - 3.15 xpfp (mr) \\
	delta   = 0.026 xfp  (mm)  - 0.009 xpfp (mr) \\
	ytar  (mm) = -.38 yfp (mm) - 0.086ypfp (mr) \\
	yptar  (mr)= 0.26 yfp (mm) - 2.1 ypfp (mr) 
\end{eqnarray}

\end{document}
